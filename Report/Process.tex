%!TEX root = Report.tex

\subsection{Process}
\label{process}

%Hvad var process planen?
%Hvordan kommunikerede vi med Harvard + vejleder?Skype
%Hvordan gik det (eval)?
%Havde vi nogen problemer med det? (tidsforskel, kultur forskel)
%Hvilke værktøjer brugte vi? (kort issue tracking)
%Virkede de efter hensigten?
%Hvilke kommunikations kanaler brugte vi?
%Virkede de efter hensigten? ville vi gerne have haft mere eller mindre kommunikation?
%Ville vi gøre noget anderledes hvis vi skulle gøre det igen/hvad har vi lært?
%Vi har sagt i god tid når vi har været på ferie og haft eksamen.

The development process in this project required more planning than usual because of the long distance collaboration with our counselor at Harvard University. This meant that we, at an early stage agreed on a time plan, a preliminary table of contents for the report and which tools we would use for communication and tracking of our progress. \\

For communication we used regular email, Skype and the Github \cite{github} wiki and issue tracking system. Email allowed us to easily communicate with our counselor at Harvard University even when either part was too busy for a Skype meeting. It also helped to express the more formal questions about the project and the email correspondance was a good reference throughout the developement of the solution and writing of this report. We made sure to update our counselors weekly via email to inform them on our progress and problems.\\

Skype enabled us to have face-to-face meetings several times in the project. A summary of each meeting was created on the wiki both for our own sake but also to make it easy for us, our counselor at ITU and our counselor at Harvard University to track our agreements and progress. The Skype meetings were more sparse than the email correspondance, since it was harder to agree on a time because of the time difference, and because they took more time which reduced the time spent coding. Since Danish culture is a high-context culture (defined by Halls \cite{halls}) Skype meetings were not greatly advantageous compared to pure email correspondance but it was nice to be able to discuss certain aspects. When a bundle of questions presented themselves at the same time it would save time to ask them in bulk. The fact that both we and our counselor at Harvard University shared the same culture (specifically work culture) made the collaboration quite a lot smoother. This made the expectations of work hours, work load and final product align more easily along with avoiding any confrontations as a result of suprising holdidays or customs etc. \\

The issue tracking system's on Github was something we started out using but quickly abandoned. Since most of our work was done in the same location with all group members present the issue tracking systems primary function was to help our counselors track our progress easily. We quickly discovered that we were using a lot of time documenting these issues and that our counselors were satisfied with the weekly email update we provided. \\

In general we were satisfied with the tools we used. The combination of verbal and written communication provided us with good channels to fulfill our needs and we feel that all of our questions were answered in a satisfactory way. Additionally we feel that our tools have helped us to inform our counselors of our progress and problems along the process in a satisfactory way. The fact that email is an asynchronus form of communication also helped us deal with the time difference between Copenhagen and Boston. We did not experience any problems with this. \\

The learning outcome of this project regarding the process has been mostly about when to change direction when encountering a technical roadblock. When we ecountered problems using JNI and communication with the PCB we should have been quicker to search for alternate solutions or abandon the cross platform design. This could have given us more time to implement more features in the final solution instead of struggling with technical problems.