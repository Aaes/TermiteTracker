\documentclass[oribibl]{llncs}
\pagestyle{headings} %page numbers
\usepackage[T1]{fontenc}
\usepackage{hyperref}
\usepackage{graphicx}
\usepackage[utf8]{inputenc} 
\usepackage{todonotes}

\usepackage[font={footnotesize}]{caption}
\captionsetup[table]{skip=10pt}

\usepackage{subcaption}
\usepackage{amsmath}

\title{TERMES: Termite tracking in collaboration with Harvard University}
\author{Nikolaj Aaes \and Niklas Schalck Johansson \and Hildur Uffe Flemberg\\
\email{\{niaa, nsjo, hufl\}@itu.dk}}
\institute{IT University of Copenhagen \linebreak Rued Langgaards Vej 7  \linebreak DK-2300 Copenhagen S}
\date{December 2013}

\makeatletter
\renewcommand*\l@author[2]{}
\renewcommand*\l@title[2]{}
\makeatletter

\begin{document}
	
\graphicspath{{img/}} %default folder for image

\maketitle

\begin{abstract}
%!TEX root = Report.tex

10 linjer \\
problem stilling \\
evaluation \\
resultat \\

• State the problem - There is no existing software that allows tracking of termites. %using the HP plotter provided by harvard as well as being able to extract relevant statistics.\\
• Say why it’s an interesting problem - It will enable biologists to make better analyses based on more empricial data \\
• Say what your solution achieves - A way to control the plotter that tracks an ant/termite in real time and collects data that can be extracted as statistics\\
• Say what follows from your solution - While this is just a prototype (in a lab enviroment) it is a step towards making new and better field equipment for biologists along with corresponding software.\\

\label{abstract}
\keywords{Computer Vision, Image Processing, Termite Tracking, Biology, Computer Science.}

\end{abstract}

\setcounter{secnumdepth}{3}
\setcounter{tocdepth}{3}
\tableofcontents
\clearpage

%er koden kommenteret?
%læs igennem for specielle ord:
% Harvard University
% xy-plotter
% analyze/analyse 
% councillor / counsillor
% ant/termite

% lad hver af kapitlerne have en intro og en del-konklusion
% hæv abstraktion
% "vision based insekt tracking"
% 1,2,3 til et kapitel
% Camera feedback / robot control -> højt abstraktion
% rediger titler -> fortæl historie
% i tracking kan vi nævne at der findes andre ting -> her
% under hver "CV" teori vil vi gerne nævne hvad man opnår i forhold til projektet
% send inden/på torsdag

%!TEX root = Report.tex

\section{Introduction}

Many species of termites, including ants, work in swarms to build structures that are many times larger than themselves. The fact that they do this in an efficient manner makes termites an interesting subject for study because we, as humans, can hope to be able to learn from their techniques and apply them in a human context in construction of houses, bridges etc. Studying termite behavior, however, requires a lab environment facilitating appropriate hard- and software tools. 

The goal of the Self-Organizing Systems Research Group (SOSRG) at Harvard University is to develop a swarm construction system in which robots cooperate to build 3D structures much larger than themselves. As of this writing, the system consists of simple robots that are able to move around and cooperate in and combining physical objects in order to build larger user defined pieces from high level descriptions. 

This project is conducted at the IT University of Copenhagen (ITU) in the fall 2013 in cooperation with SOSRG. The goal of the project is to build software to interface with Harvard hardware in order to create a stand-alone environment for tracking and monitoring ants. Specifically, 

%!TEX root = Report.tex

\section{Project Proposal}

%The Self-Organizing System Research Lab on Harvard has created a autonomous robot for tracking African termites in a lab environment. Working with and tracking these termites is cumbersome as they only thrive in environments that resemble their native environment and only when they are together with other termites. Therefore automatic tracking of specific termites is necessary, as tracking up until now has been done manually. Hardware has been created to handle this, however it lacks proper software support. \\

The purpose of this project is to develop software, such that the provided hardware can be used to track these ants/termites and analyze the output, as well as providing basic user input to the robot tracker. \\

There are three parts in this project:  

\begin{enumerate}
\item Tracking termites using a low resolution camera. In order to track a termite, it is necessary to be able to determine its position and move the camera to its new position using image analysis on the camera output. 

\item Interact with the tracking device and update the camera's position with the result from the tracking software. 

\item Design and develop a user interface that can be used by biologists to retrieve statistical data from the tracking.  
\end{enumerate}

\section{Vision based insect tracking}

%!TEX root = Report.tex

\section{Background}

%Projektet er erstatning for GSD hvad betyder det for os? (ved ikke om det skal med)
%Hvilket udstyr havde vi at arbejde med?
%Umiddlebart ville vi gerne have logic code i C++ og GUI i Java så vidt muligt. Hvilke fordele giver det os? hvorfor?
%Hvad for nogen myrer havde vi og var der nogen udfordringer her? (tænk: hvis jeg skal gøre det igen, hvad skal jeg så huske om myrer)
%Hvordan var det med at male myrerne?

This project was undertaken as a substitute for the "Global Software Development" course at the IT University of Copenhagen and therefore has additional emphasis on the international collaboration and developement process. Section XX Process will describe the tools used as well as evaluate the overall process. \\ 

Our councillor at Harvard wanted the tracking and the interfacing with the tracking device to be implemented in C++ to make it easier for further developement. The graphical user interface was not restricted when choosing programming langauge and we ended up using C\# and Windows Forms to implement it.

As mentioned in the project proposal the project was divided into three parts. The first part of the project was to be able to track termites. Since african termites are hard to come by and transport we settled for regular ants instead. While these are not as big as the termites, they behave in a similar way and the solution should be able to easily adapt to the termites. In the first part we started by implementing the tracking on a video. We received a video from Harvard of a single ant running around in a petri dish with a white background. The ant itself was painted red and green to make tracking easier. This was the simplest setup we could think of and acted as a good starting point. By recommendation we decided to use the OpenCV (INDSÆT REF) framework to implement our tracking. We will return to this framework in Section XX Framework. \\

The second part of the project interacting with our tracking device and conform the tracking code to work with the device. We received a Hewlett-Packard (INDSÆT MODEL) XY-plotter by mail from our councillor at Harvard. Additionally we received a small ant farm via mail and we bought some brushes and acryllic paint for the ants. While we discussed using infrared paint or fluorescend paint, both these ideas were discarded because both types of paint contains compounds that the termites and ants really like which makes them eat the painted ant. Along with the plotter we also received a small circuit board that plugged into the serial port of the plotter and had a USB cabel for us to communicate with it. Lastly the package contained a collection of petri dishes for us to use. \\

In this part we had to paint the ants with the paint we bought and we want to give some friendly advice to anyone who wants to do any projects involving ants. If you put them 2-4 minutes in the freezer they become very docile and much easier to paint without harming the ants. When they are taken out they act completely normal after 2-3 minutes. The ants also had a tendency to crawl up the sides of the petri dish. We were recommended to try putting teflon tape (should be really slippery) on the sides without any luck. We were also recommended to try mineral oil but we were unable to get any. Our solution was to place a small petri dish in a bigger petri dish and fill the gap with water. The ants would quickly discover that there was water surronding them which prevented them from escaping the small petri dish. \\

The third part of the project involved constructing the graphical user interface, hooking it up to the tracking code and extracting statistics.  \\

TODO: beskriv third part, indsæt refs.

\subsection{Assumptions}

TODO

\subsection{Scope}

%Kun test i et lab environment
%test på myrer, regner med at det kan generaliseres til termitter (andre størrelser)
%Kun test med de farver maling vi har
%Kun test med denne plotter
%Kontrollerede lysforhold
%Tracker kun een myre ad gangen

While the project lays the foundation for a fully useable field solution we have chosen to narrow our scope to the essential functionality. This also means that the project was designed and tested in a controlled lab environment where factors such as lighting, wind etc. are constant. \\

Because actual termites were out our reach and that our councillor at Harvard recommended it, we did all of our testing with normal ants instead of termites. We are confident that only minor adjustments are necesarry to make the solution work with termites since the main difference is the size. \\

Painting the ants in different colors can have some impact in the quality of the tracking. As a rule of thumb it is easier to track colors that contrast the background the most. In this project we chose to paint the ants white and bright green. We did not try any other colors because we found these to have a high degree of contrast which were sufficient. \\

As the HP INDSÆT MODEL plotter was the only hardware available to us we have restricted the project to only include this type of plotter. One could potentially gain better results with a plotter with a finer granularity in coordinates and smoother movement. \\

While it could be intereseting to track several ants we have chosen to focus on tracking a single ant in this project due to time restrictions. However, we do not believe that there are any technical barriers against this and would simply require more time to implement.

\subsection{Requirements}

Maa jeg foreslaa at i laver 2 modes (man kan f.eks. vaelge mode i menuen
af toppen af programmet).
1. Den ene er kalibrerings mode hvor man kan se det direkte feed oeverst
til venstre, samt billedbehandlet billede oeverst til hoejre, og sliders
nedenfor.
2. Nummer to er tracking mode hvor man kan see feedet fra kameraet taet
paa oeverst til venstre, feedet fra overview-kameraet oeverst til hoejre,
og saa statistik/grafer nederst til venstre, og valgmuligheder mht til
statistik og hvor filerne skal gemmes mv. nederst til hoejre? + start
knap.
3. Hvis i faar tid kan i tilfoeje en bias-mode hvor man kan klikke paa et
punkt i skaalen som plotteren skal forsoege at traekke myren imod (ved
hjaelp af mad).

1. Position i skaalen over tid (marker dens rute). + indstilling for hvor
lang tid tilbage ruten skal vises.
1. Gennemsnitshastighed/tid (siden start), +indstilling for running average.
2. Heatmap over hvor myren opholder sig mest i skaalen, (groft firkantet
grid er fint til at starte med).
3. Heatmap over hvor myren holder pauser i skaalen, +indstilling for hvor
lang en 'pause' er, hvor lav den gennemsnitshastighed skal vaere.
3. Tael andre myrer den moeder/tid. +Indstilling for afstand af et 'moede'.
4. Hvor lang tid der gaar imellem den moeder andre/tid (siden start).
5. Hvor meget tid den pauser naer andre myrer/tid. +Indstilling for
afstand+pause laengde
6. Hvor meget af skaalen den har undersoegt (Additivt) over tid.
+Indstilling for radius af hovedet, (hvor langt fra center af hovedet den
kan  undersoege af gangen)
7. Hvor meget nyt areal den daekker hvert interval over tid. +instilling
for interval: f.eks. 30s/1min/2min
8. Mean free path. (maal hvor lang tid der gaar mellem pauser. Pauserne
defineres af lav gennemsnitshastighed)

Mht gui'en er et mere objekt-orienteret sprog fint med mig. Saa laenge at
selve data- og billed behandlingen foregaar i C/C++.
Nedenfor har jeg forsoegt at beskrive projektet lidt mere specifikt. Hvis
i stadig er interesseret i projektet synes jeg at vi skal tage en
skype-snak om helt noejagtige instruktioner.

1. UI:
Brugervenligt, visning af live video, samt statistikker for opsamlet data.
Bruger input til manuel styring af plotteren, aendring af billed
behandlings-parametre i tilfaelde af at udstyret skal bruges til andre
insekter i fremtiden, samt input af punkter i arenaen plotteren skal
forsoege at styre imod, eller undgaa.

2. Tracking:
Formentlig den stoerste udfordring. Plotteren skal kunne tracke en termit
maerket med roed og groen maling ud fra et lav resolutions-kamera monteret
paa manipulatoren. (Hvis i finder et bedre alternativ til det nuvaerende
kamera, kan vi sagtens koebe et nyt). Termitten er hvid og brun, og dens
baggrund roed/brun jord, derudover vil den vaere omgivet af mange andre
(u-maerkede) termitter. De bevaeger sig op til 2cm/s og manipulatoren skal
helst bevares indenfor 0.5cm radius af hovedet (uden at ramme den eller
andre termitter).
Jeg vil foreslaa at i starter med at tracke myrer (de er nemmest at
finde), paa hvid baggrund, og derefter paa jord.

3. Styring af manipulator:
Manipulatorens position skal aendres afhaengigt af outputtet fra
tracking-softwaret. Hardwaret kan modtage serielle kommandoer om ny
oensket position. Hvis i oensker adgang til softwaren paa
microcontrolleren som styrer plotteren kan det ogsaa sagtens arrangeres.
Det kan blive noedvendigt at indfoere soft-start hvis plotteren skal
bevaeges langt for at undgaa pludselige ryk i maskinen der kan skraemme
insekterne.

5. Data behandling: Dette vil involvere mange maader at behandle den
indsamlede data paa. F.eks. beregning af hastighed/tid,
mean-free-path/tid, histogram af position i arenaen/tid, maengde af
interaktion/tid, mm. Jeg sender gerne en laengere liste hvis i er
interesserede.



%!TEX root = Report.tex

\section{Tracking}
This chapter is divided into three subparts. First we will introduce the reader to different image processesing techniques, and their uses. Following this we will introduce the reader to a framework that supports these techniques, and finish off by describing the implemened solution and the design choices.

% Gå ud fra at dette afsnit ikke har noget video input fra et kamera. Integration med kameraet kommer senere.
% 
% Teori først - hvilken slags image manipulation har vi brugt? hvad var alternativerne?
% Så framework - vi bruger OpenCV, why? hvad giver det os? hvad er drawbacks?
% så implementation - Hvad har vi implementeret? Hvordan var performance? hvad var vores alternativer? har vi eksperiementeret med nogen af de andre muligheder?
% 
% Image Segmentation - hvad giver det os/hvorfor er det smart?
% Thresholding
% Dilating
% Eroding
% Background detection and why we can't use it
% Alternatives and why we don't use them

\subsection{Theory}
This section will provide information about five common image processing techniques; thresholding, dilating and eroding, contrast, image segmentation and background filtering. The description of each technique will be backed up by examples. We will also use the \textit{ternary if} statement notation in this section which looks like the one shown in Equation \ref{eq:notation}.

\begin{equation}
value = {Boolean\mbox{-}Condition} ? {True\mbox{-}Evaluation}: {False\mbox{-}Evaluation}
\label{eq:notation}
\end{equation}

It works just like you would expect from many programming languages. It is also known as an inline \textit{if-statement}. \\

\noindent \textbf{Thresholding} \par
Thresholding is an image processing technique used to make a final decision about each pixel in an image. Either a pixel value is one that we are interested in or it is not. This is usually done by assigning a specific pixel value to the pixel we want, and another to those that we do not want. In general we compare the \textit{i}th pixel of the source image, \textit{src}, to the threshold value, \textit{T}, and saves the result in a destination image, \textit{dst}. For instance, to create a binary image where we are interested in all pixels above the threshold \textit{T}, the equation would look like the one shown in Equation \ref{eq:binarythresholding},

\begin{equation}
dst_i = src_i \geq T ? 255: 0
\label{eq:binarythresholding}
\end{equation}

Most threshold operations are applied on grayscale images, where all pixel values range between 0 (black) and 255 (white). Sometimes these values are normalized to range between 0 and 1 instead. However for the rest of this report we will assume that grayscale images use the former convention. Soon we will argue how thresholding can be expanded to also cover thresholding an RGB image. Figure \ref{fig:threshold_example} show an example of applying threshold to a grayscale image.

\begin{figure}
        \centering
        \begin{subfigure}[b]{0.3\textwidth}
                \includegraphics[scale = 0.2]{img/globe}
                \caption{Grayscale image}
        \end{subfigure}
		\quad
        \begin{subfigure}[b]{0.3\textwidth}
                \includegraphics[scale = 0.2]{img/post_threshold}
                \caption{Thresholded image}
        \end{subfigure}
		\caption{An example of applying a threshold on a grayscale image. This example use the threshold value \textit{T} = 200}
		\label{fig:threshold_example}
\end{figure}

Now what happens if what you are interested in is a color? To use standard thresholding we first need to convert it to a grayscale image. However doing so might result in an image that have lost important information as can be seen in figure \ref{fig:RGB2GRAY}. Unless you want to find the red color, you have no way of differentiating the blue and green color.

\begin{figure}
        \centering
        \begin{subfigure}[b]{0.3\textwidth}
                \includegraphics[scale=0.5]{img/RGB}
                \caption{RGB image}
        \end{subfigure}
		\quad
        \begin{subfigure}[b]{0.3\textwidth}
                \includegraphics[scale=0.5]{img/GrayRGB}
                \caption{Grayscale image}
        \end{subfigure}
		\caption{Example of grayscaling a color image}
		\label{fig:RGB2GRAY}
\end{figure}

A step in the right direction would be to define the threshold value \textit{T} as a scalar consisting of three values - one for each color channel. We will denote this threshold scalar as \textit{S} and define it as shown in Equation \ref{eq:thresholdscalar}.

\begin{equation}
S =  
\begin{pmatrix}
  S_{R}\\
  S_{G}\\
  S_{B}\\
\end{pmatrix}
\label{eq:thresholdscalar}
\end{equation}

To apply it to an RGB image we need to change Equation \ref{eq:binarythresholding} to the one shown in Equation \ref{eq:threshold_RGB}.

\begin{equation}
{dst_i} = {src_i}_R \geq S_R \wedge {src_i}_G \geq S_G \wedge {src_i}_B \geq S_B? 255: 0
\label{eq:threshold_RGB}
\end{equation}

Trying it out makes it much easier to differentiate between colors. In Figure \ref{fig:RGB_Thresh} a threshold attemp using this method directly on the RGB image is shown.

\begin{figure}
        \centering
        \begin{subfigure}[b]{0.3\textwidth}
                \includegraphics[scale=0.5]{img/RGB}
                \caption{RGB image}
        \end{subfigure}
		\quad
        \begin{subfigure}[b]{0.3\textwidth}
                \includegraphics[scale=0.5]{img/RGBThresh}
                \caption{Thresholded image}
        \end{subfigure}
		\caption{Example of thresholding a color image with Equation \ref{eq:threshold_RGB} using the scalar \textit{S}=(0,100,0)}
		\label{fig:RGB_Thresh}
\end{figure}

However one issue remains - what if you want to find a color among similar colors? The problem is illustrated in Figure \ref{fig:green_fail} using the same scalar as in Figure \ref{fig:RGB_Thresh}

\begin{figure}
        \centering
        \begin{subfigure}[b]{0.3\textwidth}
                \includegraphics[scale=0.5]{img/green}
                \caption{RGB image}
        \end{subfigure}
		\quad
        \begin{subfigure}[b]{0.3\textwidth}
                \includegraphics[scale=0.5]{img/simpleRGBThresh}
                \caption{Thresholded image}
        \end{subfigure}
		\caption{Example of thresholding a color image with Equation \ref{eq:threshold_RGB} using the scalar \textit{S}=(0,100,0). Unlike before the result is unsatisfactory.}
		\label{fig:green_fail}
\end{figure}

The solution is to specify a \textit{range} of acceptable values instead of just a threshold. We will specify two scalars; S Upper, \textit{SU}, and S Lower, \textit{SL}.

\begin{equation}
SL =  
\begin{pmatrix}
  SL_{R}\\
  SL_{G}\\
  SL_{B}\\
\end{pmatrix}
\label{eq:thresholdlower}
\end{equation}

\begin{equation}
SU =  
\begin{pmatrix}
  SU_{R}\\
  SU_{G}\\
  SU_{B}\\
\end{pmatrix}
\label{eq:thresholdupper}
\end{equation}

We will use the definitions in Equations \ref{eq:thresholdlower} and \ref{eq:thresholdupper} to update Equation \ref{eq:threshold_RGB}. The changes can be seen in Equation \ref{eq:threshold_range}.

\begin{equation}
{dst_i} = SU_R \geq {src_i}_R \geq SL_R \wedge SU_G \geq {src_i}_G \geq SL_G \wedge SU_B \geq {src_i}_B \geq SL_B? 255: 0
\label{eq:threshold_range}
\end{equation}

Using a range to specify a color instead will yield a much more satisfactory result as shown in Figure \ref{fig:green_final}. \\

\begin{figure}
        \centering
        \begin{subfigure}[b]{0.3\textwidth}
                \includegraphics[scale=0.5]{img/green}
                \caption{RGB image}
        \end{subfigure}
		\quad
        \begin{subfigure}[b]{0.3\textwidth}
                \includegraphics[scale=0.5]{img/finalthresh}
                \caption{Thresholded image}
        \end{subfigure}
		\caption{Example of thresholding a color image with Equation \ref{eq:threshold_range} using the range \textit{SL}=(0,100,0) and \textit{SU}=(1,150,10). We have sucessfully located the dark green color area.}
		\label{fig:green_final}
\end{figure}

\noindent \textbf{Dilating and Eroding} \par
Forklar at vi har en kernel, B, med størrelsen s, der bliver anvendt på hver pixel, p, i et billede A.
Angiv at dilating finder local optima, og at eroding finder local minima.
Angiv at de her metoder bliver brugt EFTER thresholding for at "rense" billedet. \\

\begin{figure}[!ht]
        \centering
        \begin{subfigure}[b]{0.3\textwidth}
                \includegraphics[scale = 0.2]{img/globe}
                \caption{Grayscale image}
        \end{subfigure}
		\quad
        \begin{subfigure}[b]{0.3\textwidth}
                \includegraphics[scale = 0.2]{img/erode}
                \caption{Eroded image}
        \end{subfigure}
		\quad
        \begin{subfigure}[b]{0.3\textwidth}
                \includegraphics[scale = 0.2]{img/dilate}
                \caption{Dilated image}
        \end{subfigure}
		\caption{Eroding and dilating an image. It is clear that eroding expands dark areas and dilate expands bright areas. This example uses a 7x7 kernel.}
		\label{fig:erodedilate}
\end{figure}

\newpage

\noindent \textbf{Contrast} \par
Describe how $\alpha$ and $\beta$ is applied to each pixel in an image. \\

\noindent \textbf{Image Segmentation} \par
Describe how clustering algorithms is applied to create "pixel" blobs. \\

\noindent \textbf{Background Filtering} \par
Describe how computing the absolute difference in pixel values between two image filters out the background. \\

\subsection{Framework}
Describe OpenCV. Explain why it is preferable over other frameworks. \\

\subsection{Realization}
Describe the solution and design choices. E.g. why do we use thresholding but not Background Filtering. Something about blob detection?\\




\section{Camera feedback and robot control}

%!TEX root = Report.tex

\section{Camera and Plotter Integration}
The second part of the project interacting with our tracking device and conform the tracking code to work with the device. We received a Hewlett-Packard 7046a XY-plotter by mail from our councillor at Harvard. Additionally we received a small ant farm, containing one queen and seven worker ants (no soldier ants), via mail and we bought some brushes and acryllic paint for the ants. The ants did not proliferate during the project. While we discussed using infrared paint or fluorescend paint, both these ideas were discarded because both types of paint contains compounds that the termites and ants really like which makes them eat the painted ant. Along with the plotter we also received a small circuit board that plugged into the serial port of the plotter and had a USB cabel for us to communicate with it. Lastly the package contained a collection of petri dishes for us to use. \\

In Section \ref{tracking}, we introduced the theory behind tracking of objects in images. In this section, we describe how the PC, plotter and cameras cooperate, how we get our hands on a frame to process and how we use the techniques from Section \ref{tracking} to turn an ordinary camera into a tracking device.

\subsection{Setup}

The system as a whole contains 4 components

\begin{itemize}
  \item Hewlett Packard 7046A X-Y Recorder (An XY Plotter)
  \item Printed Circuit Board (PCB)
  \item A Gigaware Web Camera 640x480 pixels (overhead camera)
  \item Web Camera 640x480 pixels (mobile camera)
\end{itemize}

\subsubsection{Plotter}
The plotter is a 45x45 table with a mechanical control panel on the side (see fig. XX). The plotter has a movable arm, moving along the X axis, consisting of a 30x3 cm of metal stretching all the way across the table. On this, a piece of plastic is attached, able to move up and down the arm along the Y axis. Together, these units cover the entire area of the plotter table. \\

The control panel is divided into labeled regions with sets of controls to manipulate specific parts of the plotter. Specifically, there is a region with controls for adjusting the zero point of the X-axis, one for the Y-axis, one for power supply etc. Most important of these is the two buttons that is used to adjust the zero point (0,0), as the software produced in this project will send only positive X and Y coordinates. Thus, the zero point must be located outside the area of the arena in order for the plotter arm to reach any possible location of the ant. \\

In order to connect the plotter to a PC, it provides an old-fashioned LPT parallel port interface. In this project, we use a printed circuit board (PCB) converting between LPT and USB. The board has a diode that will light up when signals arrive at the USB side of the board. The diode will give a constant blue light in case of an error, but will otherwise flash green. Together with a PC and an RS232 terminal like e.g. Termite from CompuPhase, this diode is useful for troubleshooting in case something is not working when running the TERMES software. 

%In order to establish a connection to the plotter for debugging purposes, use the connection parameters listed in table \ref{table:connparam}
%
%\begin{table}[h]
%\centering
%\renewcommand{\arraystretch}{1.1}
%\setlength{\tabcolsep}{8pt}
%	\begin{tabular}{ |l|l| }
% 	\hline
%  	\multicolumn{2}{|c|}{Connection Parameters} \\
% 	\hline
%  	Baud Rate & 38400 bps \\
%  	Data Bits & 8 \\
% 	Stop Bits & 1 \\
%  	Parity & none \\
% 	Flow Control & none \\
%  	\hline
%	\end{tabular}
%	\caption{Connection Parameters}
%	\label{table:connparam}
%\end{table}

\subsubsection{Cameras}
To track the ants present in the petri dish, the plotter is equipped with two low resolution standard web cameras connected directly to the PC through USB. One of the cameras are strapped on a piece of hard plastic holding the lens face-down, thus monitoring most of the plotter table from a height of approximately 30 cm. This camera is referred to as the overhead camera and is supposed to provide the user with the big overview, as well as a canvas to draw statistical overlays on, like heatmaps etc. The second camera is mounted directly on the plotter arm and follows the ant as it moves around in the arena. We will refer to this camera as the mobile camera. \\

Both cameras have a resolution of 640x480 pixels (less than 1 megapixel) as opposed to modern digital cameras with a resolution of more than 15 megapixels. Choosing relatively low resolution cameras has been a deliberate decision, as processing is done on a per-frame basis, meaning there is good reason to believe that a higher pixel density would increase processing time and thus prevent the camera from keeping up with the ant. On the other hand, the resolution should be high enough for the software to be able to identify the ant and/or the painted marker on its body. Fortunately, tests have shown that the mobile camera used in this project has been close to optimal with regards to image quality. \\

Another performance indicator for the mobile camera is the amount of frames it is able to record per second (FPS). To be able to maintain a stable tracking process, it is a key property that the ant is present in any two consecutive frames recorded by the camera. If this property is not held, it means that the ant is ahead of the camera, and the plotter will need to initiate a special procedure to relocate the ant. The camera needs at least as many FPS as needed for this to hold. On the other hand, more frames means more processing, which is also not good, but as long as the image quality is modestly low, this should not be a problem. Special techniques have been used to improve processing time, which we will discuss in Section XX. The specific mobile camera that was used in this project has a suitable quality, but could use a higher FPS value. \\

On a side note, it should be noted that the plotter is not new; the internal mechanics are sensitive and will some times cause the arm to do a fast wiggle. The effect of this is similar to taking a picture of a moving object with a DSLR camera whose shutter time is too large. The frame will be blurred and the pixels will "drag lines" that may introduce blobs similar to the one representing the ant. Thus, the plotter is in danger of being biased in a wrong direction and loose track of the ant. This problem is also likely to be reduced by using a camera with a larger FPS rate - or a newer plotter. \\

\subsection{Communication with Plotter}
In order to manipulate the plotter, a high level C++ API was developed, offering instructions like "Go to coordinate (x, y)", "move 10 units to the left", where left is defined in terms of the camera view, etc. Since the coordinate system axes of the plotter and the mobile camera grows in different directions, such an API is very convenient to have in order to construct the logic that moves the camera when the ant leaves the center of the frame. \\

In order to implement this API, we needed to know the hardware protocol of the plotter. This was, however only given informally. Basically, the plotter accepts commands in the pattern given in Equation \ref{eq:plotterformat}.

\begin{center}
  \begin{equation}
  \label{eq:plotterformat}
    0x01 +  xxxx +  yyyy
  \end{equation}
\end{center}

In this format, \textit{xxxx} and \textit{yyyy} are two 4-digit hexadecimal numbers representing the X and Y part of the target coordinate. 0x01 indicates that we are sending something. Thus the plotter can receive coordinates between (0x0, 0x0) and (0xFFFF, 0xFFFF), however, any coordinate larger than (COORDINATES MISSING) lies outside the table area and will overflow and come back at zero. \\

To transmit the coordinates through the USB connection, we use a light weight open source C library (INSERT REFERENCE). This library is implemented directly on top of the operating system specific calls for manipulating I/O resources (\texttt{WriteFile()} in Windows and \texttt{write()} in Unix), thus treating the plotter like it was manipulating a file descriptor. The library contains procedures for opening and closing the connection as well as transferring data. When sending commands in the format of Equation \ref{eq:plotterformat}, one should note that both the X and the Y part needs to be split in two parts of two hex-numbers each. Thus, the data passed to the library should be a 5-entry array of unsigned chars where the 0x01 part goes into the first entry. \\

Using this library, we developed a high level plotter interface that takes coordinates in decimal format and converts them to appropriate arrays that are sent over the connection. Furthermore, we added functions for moving the camera relatively to its own position by remembering the last requested coordinate.

%Hvilken plotter er det og hvilket udstyr sidder på den
%Hvordan er den sat til computeren
%hvad vil vi gerne have den til (nævn at det ville have været nice at have det cross platform)
%hvordan gør vi dette
%hvilke "services" ender vi med at udstille til resten af programmet (flyt til koordinat, current coordinate?)
%Hvordan er performance? kan videoen køre i real time? hvis nej hvad betyder det for os? skal vi skippe frames?

%*   0x01 = send coordinate to plotter
% *   0x01 + 2 bytes for x + 2 bytes for y (up to 10 bits) up to 0x03FF (y probably only up to 01F0)
% *   0x02 in response = sucesss
% *   0xFE in response = failure
 
\subsection{Moving the camera}
Now that we know how to move the plotter, how do we move the camera when the ant moves?
what about the volume of the movement sound?
what about the speed of the movement?


%!TEX root = Report.tex

\section{Graphical user interface}

What was the reqs for the GUI?
How did we want it to look?
What tasks do we expect the users to do (normal work flow)?
Are we satisfied with the GUI (eval)?
JNI skulle være cross platform men fungerede ikke med OpenCV.
(nævn at det ville have været nice at have det cross platform)
Vi har ikke noget der finder myren til at starte med. Hvordan kunne dette gøres?

\begin{figure}[!ht]
    \centering
    \includegraphics[scale = 0.3]{img/termes_gui.png}
    \caption{Initial GUI draft}
\end{figure}

\subsection{Statistics}
What statistics do we extract?
How do the users do this?
How are they produced?
Can they be improved (future work)?


\section{Testing, evaluation and conclusion}

%!TEX root = Report.tex

\section{Process}

TODO

\subsection{International collaboration}

TODO

\subsection{Tools}

TODO

%!TEX root = Report.tex

\subsection{Reflection}
In this section we will discuss and reflect upon issues that we believe have had a potential influence on the outcome of the project. These include a discussion of how the situation would have looked if we could have removed the requirement of using a mobile camera and how the project could have benefitted from using a different plotter. \\

Using a mobile camera is a requirement because users would potentially like to be able to stimulate the ant using food or a pheromone stick attached to the camera. However, excluding this requirement, we have no reason to believe after this project that it should not be possible to reach the same goals using only the overhead camera. This would of course require an overhead camera with a suitable resolution and tests to verify that frames with the chosen resolution can be processed in a satisfactory amount of time. The relation between the resolution of the current cameras and processing times leads us to believe that it would indeed be possible to increase resolution without increasing the processing time to an unacceptable level. However, it requires testing to know exactly how much. \\

On the positive side, using only an overhead camera would get rid of the mechanical wiggles that cause the frames to become blurry, remove sounds that might affect the behavior of the ant and remove shadows from the mobile camera on the plotter table. Furthermore, it will increase (but not remove) the upper bound we need to put on processing time per frame as there is no mobile camera to lose track of the ant between frames. With only an overhead camera, it would even be possible to do tracking based on   video files instead of a live camera input and thus eliminate all bounds on processing time (because all frames could be recorded beforehand). \\

Ignoring reflections from the water could be a useful addition to the software. Whenever a blob would be detected, we could convert its pixel position to a camera coordinate, and based on this, measure the distance to the center of the petri dish. If the distance is greater than the radius of the petri dish the blob must be a reflection from the water and can be safely ignored. A corner case where this would fail is if the ant is close to the edge, causing the reflection to merge with the ant and create a single blob where ant and reflection is indistinguishable.\\

Reducing the sounds from the plotter is also something we could have achieved by exchanging the plotter for a different model. Choosing a different plotter would also open possibilities for choosing a better model that has smoother moving arms.

%!TEX root = Report.tex

\subsection{Evaluation}

Overvej at kalde dette afsnit "Experimentation" eller "Testin"

vis hvornår det virker og hvornår det ikke virker. 
Se at det virker når vi regnede med at det virkede.

Hvad tester vi?
Hvordan tester vi det?
Virkede det efter hensigten?
Hvorfor/hvorfor ikke?
Er der nok testing?
Hvordan kan man lave mere testing?
Er der andre måder vi kunne have testet på? (fordele og ulemper ved det)


%!TEX root = Report.tex

\subsection{Threats to validity}

% Høj lyd / hastighed kan skræmme myrerne til at flytte sig hurtigere/mere.
% Lav resolution
% Myrer =/= termitter
% myrer bevæger sig ikke som i naturen
% temperatur + luftfugtighed + vind + lys osv.
% Masser af genskær (lysforhold)
% Very sensitive / unstable
% højere kamera
% at holde myren i skålen

\subsubsection{Threats to internal validity} \mbox{}\par
%Internal Validity is concerned with confirming that the correlation between the treatment and the outcome is indeed casual, and not accidental, or caused by some third variable that has not been observed. For example we may discover that all programmers in C were faster than programmers in Java, but forget that all the programmers in Java took the experiment very late at night, when they were tired.

While we believe our results to be purely causal there are always some threats to internal validity that questions which factors could have an impact on the results. We describe the important ones here:

\begin{description}
\item[Behavior of the ants] This project had a development period spanning six months. In this period the season changed and so did the temperature. The ants could have altered behavioral patterns, movement patterns and speed during the project essentially making the ants we tested during the last part of the project, behave completely different from the ants we started out with. \\

\item[Painting] We painted our first ant relatively early in the development process and the ants could have had some reaction to the paint. We did use acrylic paint as recommended by our counselor, and we did not observe any noticeable changes but it could still be present. \\

\item[Hardware] Getting the camera in the same start position every time is not something that can be done with 100\% accuracy. The small margin of error during positioning might have had a miniscule impact on each tracking session. \\

\item[Light] As we tried to do all tests at the same time of day with the same types of light each time, it was very hard to keep the environment completely identical. Since light does have a significant impact on the tracking, this threat to validity might be the greatest, but it is also something that can be hard to control. \\    
\end{description}

\subsubsection{Threats to external validity} \mbox{}\par
%External validity discusses how far the results are generalizable, or in other words how representative the sample of subjects and the circumstances of the experiment were, to be able to draw general conclusion. Do you expect the same results to be confirmed in somewhat modified conditions?

The outcome of this project is highly dependent on the environment around the hardware and the ants or termites. This makes it hard to replicate the results even with slight changes in things like light or camera resolution. This has the repercussion that while the software works in theory and in a controlled environment it is unlikely to work in the natural habitat of any ants or termites. The following list contains the environmental factors we consider to be important for the results:

\begin{description}
\item[Light] The tracking relies on thresholded images which can be deteriorated by different levels of light that makes the tracking more inaccurate and more likely not to find the ant. \\

\item[Reflections] Keeping a lid on the petri dish or filling the outer rim of the petri dish with water, like described in Section \ref{ants}, generates a lot of reflections. This is somewhat tied to the lighting factor and if the lights are positioned in such a way that the camera catches the reflections, it can deteriorate the thresholded image used for tracking. \\

\item[Camera resolution] The camera used had a low resolution of 640x480 pixels. While it could be nice to have a larger resolution, the low resolution enables us to process each frame fairly quickly. The threat to validity is mostly tied to performance. If a better camera is chosen one might need to skip more frames or upgrade the connected PC since each frame will have a longer processing time. \\

\item[Camera weight] The camera attached to the plotter during our development and testing was very light. This made the plotter stutter less when it moved and thereby generated less noise. Both sudden movements and loud noises can have an effect on the movements of the ants and if the camera is exchanged for a heavier one, it could amplify the noise and thereby impact the movement of the ants further. \\

\item[Movement of the ants] In our testing phase we observed that whenever we placed an ant in our petri dish the first thing it would do would be trying to escape. It would immediately seek the edges of the dish regardless of whether we had the rim filled with water or not. If the petri dish was bigger or if the ant species was different this might not have been the case. \\

\item[Temperature, wind and humidity] Ants, like any other animals, react to temperature, wind and humidity. If any of these factors are changed it could impact how the ants move and how they behave which in turn can impact the tracking. \\  

\item[Ants and termites] Ants are not termites. This fact is something of a gap in the testing and the results of the project. We did not have the opportunity to test the software with real African termites and therefore settled for ants. While our counselor assured us that they behave in a very similar way we do not have any data supporting that this project will work with real termites, only a strong indication that it should.

\end{description}

All in all we are quite certain that the outcome of this project can be quite hard to replicate. This makes the project more useful as a base for further development than an actual tool to take into the field in its current state. 


%!TEX root = Report.tex

\section{Related Work}

TODO


%!TEX root = Report.tex

\section{Future Work}
%More statistics
%Bias mode
%Support for multiple plotters
%Plotter control as a library
%Tests with real termites
%Developer terminal/log (print was is sent to the plotter and the answer + the hard tracking coordinate data)
%Color pickers i GUI

Because time was a limitid resource for our team there were features wi did not implement and ideas for additional functionality that could have been included. This section describes a short list of the possible extensions to the project we might have worked towrads given more time. \\

The first task would be to implement the optional requirements listed in section \ref{requirements} Requirements. ER DER NOGEN VANSKELIGEHEDER VED DET ELLER ER DET LIGE UD AF LANDEVEJEN? \\

When all these requirements were fulfilled it could be beneficial to support multi types of XY-plotters. The plotter we used was a little dated and to be able to switch to any XY plotter could be very practical. Of course this would also imply testing with an array of different plotters to see whether or not the same result could be produced. To support mulitple plotters one could implement plotter control as a service library to make it easy for other applications to control the plotter. This could expose an communication interface so it would be easy to switch the both the hardware and software of the plotter. \\

One of things we really wanted to do, but did not have the oppertunity to, was to test with real termites. The solution was design to easily be able to switch between different insects of different sizes and to test with real termites would be a strong indicator of how well this switch would work. \\

To help future development the implementation fo a developer console or log could be helpful. Developers could be able to manipulate the movement of the plotter through the terminal or interact with the tracking parameter. The log could save all the raw tracking coordinates to expose what data is recorded by the tracking and how the statistical data was created. \\

The graphical user interface could be enhanced by adding a color picker in addition to the RGB sliders that are currently present. Addtionally a third "Bias" mode could be added where an incentive (either food or pheromones) were attached to the camera, a specific point was selected via the GUI and the ant/termite would be led towards that point. This would open the possibility for even more statistics to be collected.

%!TEX root = Report.tex

\subsection{Project Conclusion}

In this report we have presented software for tracking ants and termites in a controlled environment using the OpenCV computer vision framework and an HP 7046A XY-plotter. Additionally we have developed a GUI to control the tracking as well as extract statistics of the ants' and termites' movement. We have tested the software and found that we were unable to track the ant in 31.13\% of the time, and that the tracking works in most cases but still have a small amount of cases where it fails, mostly due to the testing environment. \\

We argue that although the project has potential for further development, it contributes to the field in general and has potential to help biologists analyze the behavior of termites in a more precise way than before. This analysis can in turn help the field of swarm robotics to develop better collaboration algorithms for robots in the future.

%!TEX root = Report.tex

\section{Defition of terms}
TODO


\begin{thebibliography}{9}
    
		\bibitem{termite} Thiadmer Riemersma. (2013, December 2). \textit{Termite: a simple RS232 terminal} [Online]. Available: \url{http://www.compuphase.com/software_termite.htm}
        
        \bibitem{opencv} Itseez. (2013, December 2). \textit{OpenCV} [Online]. Available: \url{http://opencv.org/}
        
        \bibitem{fogn} Rune Larsen. (2013, December 2). \textit{Kæmpemyrer (Campotonus ligniperdus)} [Online]. Available: \url{http://www.fugleognatur.dk/artsbeskrivelse.asp?ArtsID=8063}
        
        \bibitem{github} Nikolaj Aaes. (2013, December 2). \textit{Aaes/TermiteTracker} [Online]. Available: \url{https://github.com/Aaes/TermiteTracker}
        
		\bibitem{halls} J. \& G. Olson. (2013, December 2). \textit{Surprises in Remote Software Development Teams from lectures} [Online]. Available: \url{http://dl.acm.org/citation.cfm?id=966804}
        
        \bibitem{termes1} Kirstin Petersen, Radhika Nagpal and Justin Werfel. (2013, December 2) \textit{TERMES: An Autonomous Robotic System for Three-Dimensional Collective Construction} [Online]. Available: \url{http://www.eecs.harvard.edu/ssr/papers/rss11-petersen.pdf}
        
        \bibitem {termes2} Kirstin Petersen, Radhika Nagpal and Justin Werfel. (2013, December 2) \textit{Distributed Multi-Robot Algorithms for the TERMES 3D Collective Construction System} [Online]. Available: \url{http://www.eecs.harvard.edu/ssr/papers/iros11wksp-werfel.pdf}
        
\end{thebibliography}

%!TEX root = Report.tex

\appendix
\section{First Appendix}

\end{document}
